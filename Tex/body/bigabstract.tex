%%==================================================
%% bigabstract.tex for SJTU Bachelor Thesis
%% version: 0.5.2
%% Encoding: UTF-8
%%==================================================

\begin{bigabstract}

Robot system has drawn more attentions since the 1960s, with the developing of control theory, sensor manufacturing and computer capability. However, applications of robot have evolved, so robots get more and more complex to maintain the necessary functionality. The complexity of robots makes it more difficult to upgrade, and for some specific application, it may not need all the functions of a robot, then there will be a great waste of resource. Therefore, the concept of modular robot has been introduced, to divide the functions into different independent parts which can be easily combined and upgraded.

Modularization is not the only requirement of robot developing, to make a robot move is another crucial research area of  the study of robots. A mobile robot not only can do a task but can also do that task in multiple positions, which can save a lot of money on buying new machines. And in some exploring applications, a mobile robot can replace human beings to get into unknown and dangerous area to help people finish the exploring tasks. 

Since modularization and mobility are two important parts of robot developing, then the combination of these parts will result a birth of a new kind of very useful robots. In this paper, the author will illustrate a design of the control system of this new kind of robots.

According to the understanding of the author, this system should contain three crucial parts. The first is the mechanism of the robot body, which is the part interacting with external environment and containing task actuators. The second part is the circuits which will be used to drive actuators. The last part is the software which provide functions and rules to control the behaviour of the robots. The control system need all these three parts to work together to achieve the functions of this newly designed robot. Details of the design of these three parts provided in the paper, and followings are some short reviews.

For the mechanism of the robot system, the paper provides several new concepts and  design of the robot body based on an old design from the author's lab. The old design contains a central control unit to control all the modules of the robot. It has been proved not efficient because with the increasing of modules, the extended ports have to be redesign and the central control unit has to be reprogrammed. However, if each module is self-controllable, then we will get rid of the trouble of reprogramme. The connection port we will use is also fixed, which would only be used for communication between different modules. Based on this kind of idea, in this paper, it provided a new design of of module body, sensor stand and communication connection port. A new design of wheel encoder will also be provide in this section.

For the circuit part, this paper introduce a new type of structure of circuit combination. Because all the modules are self-controllable, each module has to have a central control unit. However, different modules have different functions, then we have to operate functional circuit from the main control board. In the paper, the author propose a main control board plus peripherals structure to form the control circuit of each module. The main control board only provide the power management unit and module level communication circuit. All the special functions will be provided by the peripherals. The peripherals designed in this paper includes IMU unit, Motor driver circuit, encoder support circuit, sensor driver circuit and a communication port circuit for the connection part.

For the software part, the paper only discuss the software which is need for self-navigation in unknown environment. The low level programme is the driver programme for the chassis. In this part, it contains the initialization of the related register on MCU, the PID controller that make the moving direction of the robot along with the direction that has  been calculated and the free turn methods. The mid-level programme is the programme for localization. In the paper, it provides three ways to localize the robot position. The high level programme is the programme that controls the robot moving to the target point avoiding all the obstacles on the path. In addition, this paper also talked about a way to record the detected environment and calculate the optimal path.

After design of all those three parts, the author provide a simulation of the algorithm in a simulation environment called GAZEBO. In the paper, it provides enough details about the simulation process and get the desired result which can proofed the validity of the algorithm proposed in previous section.
\end{bigabstract}
