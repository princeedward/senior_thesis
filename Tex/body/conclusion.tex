%%==================================================
%% conclusion.tex for SJTU Bachelor Thesis
%% version: 0.5.2
%% Encoding: UTF-8
%%==================================================

\chapter*{全文总结\markboth{全文总结}{}}
\addcontentsline{toc}{chapter}{全文总结}

本文详细的介绍了多功能模块化机器人控制系统的设计方案,整套系统共分为3各部分。第一部分为在实际环境中进行任务执行的机械结构,其中大部分的设计继承了实验室前辈的设计方案,但对关键部位进行了改良。其中包括对每个子模块的机械结构进行了重设计,如重新设计了传感器支架和增加了轮式编码器。第二部分为对执行机构动力元件进行驱动的电路。电路按功能可以分为两类,第一类为所有模块都要使用的通用控制接口板;第二类为根据每个模块不同功能而选用的外设电路。其中通用接口板具有着控制的核心单元——单片机,和模块间通讯的关键接口——CAN总线;而外设电路则都是根据具体功能而设计的,包括了定位用IMU、编码器模块;对电机驱动的驱动模块;对传感器进行驱动的传感器接口模块以及要放置在连接处的模块间通讯接口实现模块。第三部分为控制算法部分,控制算法主要涉及四点,对底盘的驱动,对车体的定位,对障碍物的躲避,对所经历环境地图的绘制与最优路线的计算。这部分的每一个内容都是承上启下的。最后是对仿真过程进行了详细的介绍。

经过对系统三部分的研究设计出的控制系统整体,很好的满足了工业与探索应用领域对于模块化机器人的要求。本设计提供的控制方案是按功能进行划分的,只要合理规划,本方案中涉及的方法是很容易移植到其他的机器人及应用中去的。最后经过仿真的验证,证明了本设计提出的蔽障算法是实际可行的。

