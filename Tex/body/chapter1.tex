%%==========================
%% chapter1.tex for SJTU Bachelor Thesis
%% content: introduction
%% version: 0.0.1
%% Encoding: UTF-8
%%==================================================

%\bibliographystyle{sjtu2} %[此处用于每章都生产参考文献]
\chapter{绪论}
\label{chap:introduction}

\section{模块化机器人的研究背景}

随着控制理论,传感器,计算机科学和人工智能等技术的发展,机器人的研究越来越受到关注。从上世纪90年代至今,机器人技术得到了空前的发展,由单一化,大型化和功能固定化转向小型化,廉价化和模块化。与此同时,机器人技术正在被应用在越来越多的领域,从工业生产,到未知环境探测,从航天工程,到服务餐饮。而在绝大多数的应用中,多或多或少的对机器人的移动性有要求,如生产线中的搬运机器人,应用在航天探测中的火星车月球车,应用在军事领域的拆弹机器人等等。所以机器人的移动性一直是机器人研究领域的热点。

而在这个快速变化的世界里,如果人们可以对机器人进行简单的升级,就像升级软件一样,这将是具有革命意义的。模块化则正是一个很好的解决方案。模块化是根据功能将机器人进行拆分,并通过用户对于具体功能的需求进行组合的过程。用户不再为自己不需要的功能而付出任何金钱上的代价,正相反,每一个模块化机器人都像是面向用户定制的产品,而这一产品对于用户永远是最优的。同时,一个具有移动能力的机器人也会比一个只能固定在单一工作区间的机器人有更多的应用价值和节省更多的成本。这就我的研究所要关注的方向,如何将功能进行拆分,和如何对众多模块进行控制,和如何使其移动。 \\

在这一领域,已经有了一些相关研究。但是对于模块化的机器人,目前的研究主要集中在自组装,应用在特殊领域的机器人上。CMU和Standford都有很多这一研究领域的专家。而对于移动化智能机器人的研究,目光主要集中在本身的移动系统上。这些机器人往往会根据需要而去选择传感器,却不会根据传感器限制,去使用有限的感知条件去适应和感知未知环境。机器人只有对于不同场合选用不同的传感手段,这样才能做到最优化,避免了不必要的资源浪费。但现在的智能移动机器人为了兼容尽可能多的环境,而装备了多种传感器,应用了多种传感手段,这对于日常应用,是一种不必要的浪费。而模块化正可以解决这一问题。我的研究将主要集中在如何拆分多种传感手段,如何是传感信息和主控制系统沟通,如何集成多种其他功能上。而这套系统模型的建立,传输机制和协议的制定,将会提供一整套嵌入式模块的解决方案。


\section{避障式移动机器人及路径规划研究现状}



\section{课题的研究内容和意义}



