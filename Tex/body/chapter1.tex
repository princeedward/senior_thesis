%%==========================
%% chapter1.tex for SJTU Bachelor Thesis
%% content: introduction
%% version: 0.0.1
%% Encoding: UTF-8
%%==================================================

%\bibliographystyle{sjtu2} %[此处用于每章都生产参考文献]
\chapter{绪论}
\label{chap:introduction}

\section{模块化机器人的研究背景}

随着控制理论,传感器,计算机科学和人工智能等技术的发展,机器人的研究越来越受到关注。从上世纪90年代至今,机器人技术得到了空前的发展,由单一化,大型化和功能固定化转向小型化,廉价化和模块化\upcite{dudek2010computational}。与此同时,机器人技术正在被应用在越来越多的领域,从工业生产,到未知环境探测,从航天工程,到服务餐饮。而在绝大多数的应用中,或多或少的对机器人的移动性能有要求,如生产线中的搬运机器人,应用在航天探测中的火星车月球车,应用在军事领域的拆弹机器人等等。所以机器人的移动性一直是机器人研究领域的热点。如W. Grey Walter等人在1948年设计并演示的移动机器人Elmer 和 Elsie\upcite{holland1997grey}。这两个机器人可以说是自主式移动机器人的“祖先”了。时至今日,只具备简单移动能力的机器人显然无法满足现代社会对于机器人功能的需求。因此移动机器人被加装各种各样的机构而成为更为复杂的功能型机器人。

随着人们对机器人功能需求的不断增加,机器人越做越复杂。但是功能强大的机器人却未必是完成单一工作的最优选择,因为对于机器人在某一阶段所从事的工作来说,大多数其他功能并没有被使用到。在这样的背景下,模块化正是一个很好的解决方案。模块化是根据功能将机器人进行拆分,并通过用户对于具体功能的需求进行组合的过程。用户不再为自己不需要的功能而付出任何金钱上的代价,正相反,每一个模块化机器人都像是面向用户定制的产品,而这一产品对于用户永远是最优的。同时,一个具有移动能力的模块化机器人相较于其他机器人有更多的应用价值和能节省更多的成本。 

模块化机器人的研究,始于上个世纪80年代中期\upcite{harrison1987study}。而对于模块化机器人,也有了越来越准确的定义:它由多个功能模块及标准接口装配而成,各种功能模块可批量生产、独立维修、独立扩展,快速组装成不同性能的移动机器人;模块化移动机器人之间可局部通信、相互合作,完成全局任务。它有下面显著特征:更广泛的任务领域、更高的效率、改良的系统性能、容错性、鲁棒性、更低的经济成本、容易开发、分布式的感知与作用以及内在的并行性等。 

目前对于模块化机器人的研究主要集中在两个方向上,一为静态可重构机器人,另一个为自重构机器人。静态可重构机器人是指,可以借助外力进行重构的,在工作状态是其结构为静态结构的机器人\upcite{wpf2007}。简单地说,是有人为选择模块并可利用这些模块的工作的机器人。 而自重构机器人这是可以动态的改变自身结构的机器人。

在静态可重构机器人领域,已有很多人做过相应的研究。如BenhabibB,DaiM等人\upcite{benhabib1991mechanical}设计了一个基于遥驱动技术的模块机器人单元,驱动方式类似于传统的工业机器人。Paredis C J J等人\upcite{paredis1993kinematic}设计了RMMS模块化机器人系统。它有一个模块库,可根据需要来搭建模块,实现不同的功能。Fujita\upcite{fujita1999reconfigurable}在Sony公司的OPEN-R标准之下,开发了一套可重构模块化机器人系统。可以通过众多模块组成不同的机械结构。 

在自重构机器人的领域,也有很多人进行了相关研究。如M Yim\upcite{yim1994new}设计的Polypod和PolyBot系统。还有Murata\upcite{yim1994new}的机器人系统Fracta,是通过仿生学的细胞概念而提出的一种三维的自重构机器人系统,等等。

\section{避障式移动机器人及路径规划研究现状}
移动机器人的研究最早可以追溯到上个世纪60年代\upcite{ll2002mobile}。斯坦福研究院的Nils Nilssen和 Charles Rosen等人在那一时期设计出了一款叫做shakey的自动避障移动机器人\upcite{nilsson1969mobile}。随着计算机的应用和传感技术的发展,有越来越多的的公司和研究机构参与到移动机器人的研究中来,从而大大促进了移动机器人技术的发展。 

对于移动化智能机器人的研究,两大关键研究领域是定位和环境探知(传感器)。现在的机器人导航和定位有多种方法,如基于环境信息的地图模型匹配导航,基于各种导航信号的陆标导航,视觉导航和触觉导航等\upcite{ll2002mobile}。环境地图模型匹配导航机器人可以通过多种传感手段对于地图与采集信息信息进行匹配,从而得到自身所处位置信息,最后通过规划算法来行进至目的地。这是一套基于已知环境的移动方案。陆标导航则是在机器人活动区域设置可以探知的信号源,通过对信号源的感知来获得自身相对位置,并在路标的指引下,向终点前进的方法。视觉和触觉导航,这是通过视觉和触觉传感器,对行进环境进行识别, 并作出行进决策的导航手段。这是一种动态的可适用于未知环境的机器人移动方案。移动机器人传感技术主要是对机器人自身内部的位置和方向信息以及外部环境信息的检测和处理,采用的传感器分为内部传感器和外部传感器,其中内部传感器有:编码器,线加速度计,陀螺仪,磁罗盘,激光全局定位传感器,激光雷达等\upcite{ll2002mobile}。 

而基于两大模块,避障算法的研究也有了长足的发展。避障算法经历了人为设计环境避障,已知环境避障,半已知环境避障和完全未知环境避障等几个发展阶段。Ting-Kai Wang等人\upcite{wang2010path}提出了基于障碍物边界和模糊逻辑的未知环境避障算法。Torvald Ersson等人\upcite{ersson2001path}提出了基于网络化简的针对于短程传感器的未知环境探测算法。Shuichi Utsugi\upcite{utsugi2008path}提出了一种基于视觉的连续关注点捕捉技术的智能避障算法,等等。 

\section{课题的研究内容和意义}
移动机器人的功能化带来了其结构过于复杂化的问题。而对于大多数应用领域来说,如不能使用机器人的全部功能,将造成巨大的资源浪费。通过之前的讨论可以得知,模块化是解决功能复杂化的最好的方法。人们对于机器人的选用,也会由机器人能提供什么,到希望机器人能做什么的方向转变。只选用自己有用的模块而使机器人的使用和维护成本达到最小。与此同时,模块化机器人的实现,也为系统的升级减小了开支。设想人们可以对机器人进行简单的升级,就像升级软件一样,这将是具有革命意义的。因此,我对于复杂机器人的模块化研究,将主要集中在如何将功能进行拆分和如何对众多模块进行控制上。

正如先前所讨论的,定位对于移动机器人控制来说是一个非常重要的环节。而如何对移动的机器人进行定位,这是我的设计所需要面临的另一个问题。目前对于定位的研究,非常依赖于GPS模块,因为其相对简单,信息容易获取。但存在的缺点是,只能在开阔的户外环境中所使用。而对于需要在楼内或障碍物内移动的机器人来说,这一方法是极其不可靠的。所以我需要在前人的设计基础上进行一定的改进,引入一种可以获得坐标信息的机械结构到模块化底盘上。通过对虚拟坐标的计算来获得车体的位置信息,从而完成定位。

关于避障算法的研究,已经进行了多年,多种方案都已经被提出。但这些方案往往会根据需要而去选择传感器,却不会根据传感器限制,去使用有限的感知条件去适应和感知未知环境。机器人只有对于不同场合选用不同的传感手段,这样才能做到最优化,避免了不必要的资源浪费。但现在的智能移动机器人为了兼容尽可能多的环境,而装备了多种传感器,应用了多种传感手段,这对于日常应用,是一种不必要的浪费。模块化的拆分思想这有助于建立解决这一问题的思路。根据这一思想,我对于传感器和避障算法的研究将主要集中在如何拆分多种传感手段,如何使传感信息和主控制系统沟通,如何利用这一信息而完成避障和驶向目标点。而这套系统模型的建立,传输机制和协议的制定,将会提供一整套嵌入式模块移动避障机器人的解决方案。

综上所述,我的主要设计任务是设计多功能模块化移动机器人的控制系统,根据单个机器人的功能特征及多个模块间的协作行为,确定该控制系统的结构;这将包括硬件及软件层面两个部分。硬件层面,实现多机互通接口的设计,模块核心控制单元的设计。并成功实现对底盘模块的驱动,和传感器模块对外部环境的感知;软件层面,实现多模块事件响应模型的设计,标准化传输协议的制定,底盘实际驾驶控制、定位和追踪,传感器模块信号处理分析及核心控制单元功能的实现。开发仿真平台,实现单个多功能模块化机器人在未知环境下的避障移动,应用或改善现有路径规划算法。同时可以简单模拟多模块通讯协议机制。 


