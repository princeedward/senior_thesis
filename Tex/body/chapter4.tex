%%==================================================
%% chapter03.tex for SJTU Bachelor Thesis
%% version: 0.5.2
%% Encoding: UTF-8
%%==================================================

% \bibliographystyle{sjtu2} %[此处用于每章都生产参考文献]

\chapter{控制与智能避障算法}
\label{chap:algorithm}

在本章将主要介绍模块化移动机器人控制系统的软件系统。由于本移动机器人具有自主移动,自主定位,自动避障,地图生成和最优路径计算等功能,所以需要将其软件系统分为底盘电机驱动,车体定位,未知环境避障和探测地图绘制等部分。其中只有电机驱动部分是与选用的控制芯片有关系的,而其他诸部分皆是上层算法,所以处理驱动算法的程序不可移植外,其他部分皆可移植到其他系统中去。在下面的介绍中,除了会给出具体算法程序外,还会对每一部分算法的数学原理进行简单的介绍。

\section{底盘驱动算法}
底盘驱动采用了经典的H桥方案,通过单一IO口的高低电平来对方向进行控制,而在使能端接入PWM方波,通过控制占空比来控制电机的转速。控制占空比来控制转速的具体原理是,占空比的改变可以改变方波的等效电压。设一列方波的占空比是$a$,则在一个方波周期内等效电压与占空比的关系为 \\
\begin{equation}\label{eqn.equavalentVoltage}
V_{org}aT=V_{equ}T
\end{equation}
其中$V_{org}$是方波最大幅值,$T$为方波周期,$V_{equ}$为等效电压。通过公式\eqref{eqn.equavalentVoltage},可以得到\\
\begin{equation}
V_{equ} = aV_{org}
\end{equation}
等效电压与占空比呈简单线性关系。而通过电机的特性公式\eqref{eqn.motorCha} \\
\begin{equation}\label{eqn.motorCha}
V_{supply} = K_{e}\dot{\theta}
\end{equation}
可知电机转速和输入电压$V_{sipply}$成正比,这就意味轮子的转速与方波占空比成正比。所以控制轮子的转速只需要改变输入方波的占空比就可以了。但是占空比只给出了一个速度的相对关系,并没有与具体速度值的对应,所以控制的时候还要通过传感器的数据,对速度控制进行闭环。

电机系统所需输出口初始化,和一般电机控制程序如下。此段程序,是参考罗翔的程序所撰写。
\begin{lstlisting}[language={C}, caption={针对BTS7960的单片机初始化程序}]
#include "BTS7960B.h"
#include "SystemConfig.h"

/* Crystal's frequecy is 40MHz
   USE PPL change the bus frequecy*/
//---------宏参数定义-------------
#define HIGH   1
#define LOW    0
#define OUTPUT 1
#define INPUT  0

#define SETPWM23 0x0C                    //置1 PWM23选择项
#define CLRPWM23 (~SETPWM23)             //清零PWM23选择项
#define CLKB1D08 0x30                    //ClockB运行方式

#if SYS_BUS_FREQUENCY == SYSTEM_BUS_64M 
//64M-预分频: fA=2M, fB=4M(使用fB)    
    #define _PWMMAX 199                      //占空比最大值
    #define _PWMMIN 0                        //占空比最小值
    #define _PWMUNT 2    
#else                   
//其他分频: fA = 2.5MHz, fB = 5MHz,请在此基础上设定频率 
    #define _PWMMAX 249                      //占空比最大值
    #define _PWMMIN 0                        //占空比最小值 
    #define _PWMUNT 5/2   
#endif   

//---------操作函数定义-------------
void BTS7960BRestart(void)     
//重启函数:对需使用I/O口初始化
{
    DDRM_DDRM5 = OUTPUT;
    _BTSDRVE   = LOW   ;
    DDRM_DDRM4 = INPUT ;
    _BTSDRVS   = LOW   ;//Initial BITS OF PTM PORT 
    
    PWME     &= CLRPWM23;//关PWM23 
    
    //PWMPRCLK |= CLKB1D08;//ClockB frequency = 40MHz/8=5MHz
    //PWMSCLB   = 5       ;
    
    PWMPER2   = _PWMMAX ;//PWM2 frequency = 10kHz
    PWMDTY2   = _PWMMAX ;//占空比0 
    PWMPER3   = _PWMMAX ;//PWM3 frequency = 10kHz
    PWMDTY3   = _PWMMAX ;//占空比0    
    
    PWME     |= SETPWM23;
    
    BTS7960BEnable(HIGH) ;   
}

unsigned char BTS7960BControl(unsigned char ForwardPWM, unsigned char BackwardPWM)
//控制函数:正反转控制。输入PWM占空比,输出正反转状态 (输入均为百分之)
{
    unsigned char PStatus = BTS7960STOP;
    //PWM状态赋值
    PStatus = (ForwardPWM  > BackwardPWM)?BTS7960FORWARD:BTS7960BACKWARD;//电机运行状态判断
    PStatus = (ForwardPWM == BackwardPWM)?BTS7960STOP   :PStatus        ;
    
    ForwardPWM  = (ForwardPWM  > 100)?100:ForwardPWM ;
    BackwardPWM = (BackwardPWM > 100)?100:BackwardPWM;
    PWMDTY2 = (byte)((_PWMMAX < ForwardPWM  * _PWMUNT)?_PWMMIN:(_PWMMAX - ForwardPWM *_PWMUNT));//计算PWM占空比
    PWMDTY3 = (byte)((_PWMMAX < BackwardPWM * _PWMUNT)?_PWMMIN:(_PWMMAX - BackwardPWM*_PWMUNT));
    return PStatus;
}              
\end{lstlisting}
在底盘模块正常情况下是沿着算法所确定的方向匀速行进,速度只需规定一个为常数的占空比值就可以了。但是因为电机特性不一致的原因,即便规定同一PWM值,车体行进方向仍然可能偏离既定方向,所以要使底盘保持行进方向,就要对其行进角度进行闭环。这一功能的可以通过电子罗盘返回的数据来设计实现。这里将采用传统的PID算法来进行闭环程序的设计。PID系统的传递方程的拉普拉斯形式可以写为 \\
\begin{equation}
G(s)= K_p+\frac{K_i}{s}+K_ds
\end{equation}
其中$K_p$,$K_i$和$K_d$为PID方法的参数,对于系统方程,输入量是实际角度与计算角度的差值,输出量是驱动左轮与驱动右轮PWM占空比的差值。因为系统实际是离散的,所以需要将系统方程从S域映射到离散系统的Z域,关于这个控制器的C程序就是根据这一思想设计的,其伪代码可以表示如下。
\begin{lstlisting}[language={C}, caption={PID控制器伪代码}]
double KP = 一个合理的值;
double KI = 一个合理的值;
double KD = 一个合理的值;
double Error = 0;			//实际角度与计算角度的偏差
static double ErrorHis = 0;	//对于角度偏差的记录
static double ErrorAcu = 0;	//偏差积分

Error = AngleActual - AngleCalculated;
ErrorAcu += Error ;
DutyCycleDiff = KP*Error + KD*(Error-ErrorHis) + Ki*ErrorAcu ;
ErrorHis = Error;
\end{lstlisting}
关于底盘模块行进方向角度的计算将在未知环境避障算法部分介绍,关于遇到避障时转向的控制也将在那部分章节介绍。
\section{车体定位方案}
车体定位方案将直接决定了底盘模块的行进轨迹,对于是否完成目标的判断和地图的绘制等功能的实现。为了获得机器人准确的位置,在本设计中,使底盘集成了多种传感器模块。这其中包括轮式编码器,俗称拖地轮;IMU模块等单元,如果需要,则可以通过串口扩展,很容易的链接GPS模块。在下文中将详细介绍各个方案定位的数学原理,及坐标值获得的方法。

\subsection{拖地轮及方位角传感器定位}
拖地轮的机械结构设计已经在第二章的相应部分进行详细的叙述,在本节,将主要介绍基于拖地轮的定位算法。拖地轮的设计目标之一就要简化计算,所以将拖地轮固定在底盘模块相邻的两条边中央的位子上。底盘模块的坐标系,与轮子的位置如图\ref{fig.ChesisCoor} 所示。
\begin{figure}[!htp]
  \centering
  \includegraphics[width=0.5\textwidth]{chap4/ModelCordinates.PNG}
  \bicaption[fig.ChesisCoor]{底盘上坐标系的定义}{底盘上坐标系的定义}{Fig}{The Definition of the Coordinates On the Chassis}
\end{figure}

其中拖地轮a只能采集到与底盘模块y坐标轴平行的转动,同理拖地轮b只能采集到与底盘模块x坐标轴平行的转动。底盘的控制芯片将会以一定的周期进行对传感器信息的采集和对底盘模块体的控制。这一周期在本设计中为10毫秒,所以控制周期是100HZ。当每一个周期结束后,如果只有y方向的轮子采集到数据,则机器人体的世界坐标改变量为 \\
\begin{equation}
		\left\{
             \begin{array}{lcl}
            	\Delta x_{world} &= \Delta y\cdot \sin\theta \\
             	\Delta y_{world} &= \Delta y\cdot \cos\theta  
             \end{array}  
        \right.
\end{equation}
其中$x_{world}$,$y_{world}$为机器人体在世界坐标系中的坐标,$\theta$为机器人前进方向与世界坐标y轴所成的夹角。

同理当只有x方向的轮子采集到位移数据时,则机器人的世界坐标为 \\
\begin{equation}
		\left\{
             \begin{array}{lcl}
            	\Delta x_{world} &= \Delta x\cdot \cos\theta \\
             	\Delta y_{world} &= \Delta x\cdot \sin\theta  
             \end{array}  
        \right.
\end{equation}
而在同时采集到x与y方向的位移信息时,就需要用到一定的数学技巧来获得世界坐标的位移改变量。如图\ref{fig.ChesisMotion}所示,因为控制周期比较短,可以用直角模型来计算机器人体的位移,并假设车体的运动方向是初始时的运动方向保持不变,这样就可以通过机器人前进方向与世界坐标y轴的夹角和采集到的传感器信息来计算车体的实际位移了。
\begin{figure}[!htp]
  \centering
  \includegraphics[width=0.5\textwidth]{chap4/OnMotion.PNG}
  \bicaption[fig.ChesisMotion]{机器人运动示意}{机器人运动示意}{Fig}{The Motion of the Robot in World Coordinates}
\end{figure}

这时世界坐标该变量的计算公式为 \\
\begin{equation}
		\left\{
             \begin{array}{lcl}
            	\Delta x_{world} &= \Delta y\cdot \sin\theta - \Delta x\cdot \cos\theta \\
             	\Delta y_{world} &= \Delta y\cdot \cos\theta + \Delta x\cdot \sin\theta  
             \end{array}  
        \right.
\end{equation}
\subsection{IMU定位}
通过IMU上所拥有的传感器元件可知,其对于位置的确定是依赖于积分的。加速度传感器的输出量是加速度,陀螺仪的输出量是角加速度,电子罗盘的输出量是角度。表征速度,位移和加速度关系的公式是 \\
\begin{equation}
D = \int\dot{D}=\int_T\int_T\ddot{D}
\end{equation}
所以只要对加速度进行二次积分就可以算出在某一轴上的位移分量。因为控制周期比较短,所以可以认为在一个周期$T$内,机器人是在做匀加速度,匀速度运动。如果以$\ddot{x}$和$\ddot{y}$作为在底盘模块参考系中的加速度传感器所能测出的加速度量,则坐标的数学表达式可以写作 \\
\begin{equation}
		\left\{
             \begin{array}{lcl}
            	\Delta x_{world} &= \ddot{y}\cdot T^2\cdot \sin\theta - \ddot{x}\cdot T^2\cdot \cos\theta \\
             	\Delta y_{world} &= \ddot{y}\cdot T^2\cdot \cos\theta + \ddot{x}\cdot T^2\cdot \sin\theta  
             \end{array}  
        \right.
\end{equation}
但是因为加速度传感器的噪声带会对积分误差的收敛造成影响,所以光靠加速度传感器的值做计算是不够的。还需要加上陀螺仪的输出值作参考。加速度值与陀螺仪测得的关于垂直于地面平面的转轴的角加速度值的关系是 \\
\begin{equation}
\arctan(\frac{\ddot{x}\cdot T^2}{\ddot{y}\cdot T^2}) = \int_T\int_T \ddot{\phi}
\end{equation}
其中$\phi$是角加速度。

但是仅仅是使用这个比较依然是不可靠的,还要对采集到的数据进行滤波。本设计根据Houtekamer等人\upcite{houtekamer1998data} 的文章,设计了对信号处理的卡尔曼滤波器。其C程序实现如下。
\begin{lstlisting}[language={C}, caption={Kalman滤波的C程序代码}]
float angle, angle_dot; 		//外部需要引用的变量
float Q_angle = 0.05, Q_gyro = 0.01, R_angle = 5, dt = 0.005;     
float P[2][2] = {{ 1, 0 },{ 0, 1 }} ;	
float Pdot[4] ={0,0,0,0} ;
const char C_0 = 1;
float q_bias, angle_err, PCt_0, PCt_1, E, K_0, K_1, t_0, t_1;
//-------------------------------------------------------
void Kalman_Filter(float angle_m,float gyro_m)			
{
	angle+=(gyro_m-q_bias) * dt;			//先验估计
	
	Pdot[0]=Q_angle - P[0][1] - P[1][0];	//先验估计误差协方差的微分
	Pdot[1]=- P[1][1];
	Pdot[2]=- P[1][1];
	Pdot[3]=Q_gyro;
	
	P[0][0] += Pdot[0] * dt;				// 先验估计误差协方差
	P[0][1] += Pdot[1] * dt;
	P[1][0] += Pdot[2] * dt;
	P[1][1] += Pdot[3] * dt;
	
	
	angle_err = angle_m - angle;			
	
	PCt_0 = C_0 * P[0][0];
	PCt_1 = C_0 * P[1][0];
	
	E = R_angle + C_0 * PCt_0;
	
	K_0 = PCt_0 / E;//Kk
	K_1 = PCt_1 / E;
	
	t_0 = PCt_0;
	t_1 = C_0 * P[0][1];

	P[0][0] -= K_0 * t_0;					//后验估计误差协方差
	P[0][1] -= K_0 * t_1;
	P[1][0] -= K_1 * t_0;
	P[1][1] -= K_1 * t_1;
	
	angle	+= K_0 * angle_err;				//后验估计
	q_bias	+= K_1 * angle_err;				//后验估计
	angle_dot = gyro_m-q_bias;				//输出值(后验估计)的微分 = 角速度
}
\end{lstlisting}
\section{未知环境避障算法}
在本设计中采用了虚拟势能场的算法来实现路径计算和自动避障的。这一算法最先由Khatib \upcite{khatib1986real}提出,并应用于机械臂行进路线的规划。后来虚拟势场的概念经由Yoram Koren等人\upcite{koren1991potential} 的发展,而可以应用到平面移动物体的二维运动规划中。

基本的虚拟势场思想主要是引入了两个虚拟力。一个为虚拟引力,一个为虚拟斥力。虚拟引力是指运动体与目标点之间的力,虚拟斥力是指运动体与障碍物之间的力。目标点在全局坐标系中的位置是已知的,所以在初始状态下就存在着引力场的作用,是机器人向目标点运动。因为势能的等势线为圆形,所以只有运动物体向着目标点前进,才是势能变化梯度最大方向。因此在某种意义上讲,由虚拟引力所计算出的运动方向是最优的。在本设计中,设计的机器人所处的环境是未知的,就是说初始条件下并无障碍物的存在,而在运动过程中会遇到障碍物,并在障碍物的斥力影响下而改变运动方向。这说明斥力是一种短程力,它的作用范围是有限的。

在本设计中接纳了耿兆丰等人\upcite{gzf1992VPpath} 的方案,将目标点产生的势场$P$定义为 \\
\begin{equation}
P = r^2/2 = [(x-x_t)^2+(y-y_t)^2]/2
\end{equation}
其中$x_t$,$y_t$为目标点坐标。则此时机器人所处位置的势场梯度可以表示为 \\
\begin{equation}
\nabla P = \frac{\partial P}{\partial x}i+\frac{\partial P}{\partial y}j = (x-x_t)i+(y-y_t)
\end{equation}
因此引力可以表示为 \\
\begin{equation}
F_{att} = K(|\nabla P|_{r=d})=Kd
\end{equation}
其中$K$为引力常数。

对于斥力,正如上文所说,为短程力,所以其应该具有随着距离的变长而减弱的特点,如果斥力能与距离成反比的话,将符合这一要求。所以斥力的形式应为 \\
\begin{equation}
F_{rep} = \frac{K_r}{(x-x_o)^2+(y-y_o)^2}
\end{equation}
其中$K_r$为斥力常数。与引力不同,斥力是一个三维力,计算斥力时不仅相对于质心坐标,而是相对于传感器所在的点,所以$x_0$和$y_0$是每个传感器点的坐标。在此处分母使用距离的平方是因为,第一可以简化计算,提高计算速度;第二可以加快斥力的衰减和增加。斥力的作用方向即为力场梯度方向。 \\
\begin{equation}
\nabla F_{rep} = -2\frac{K_r(x-x_o)}{((x-x_o)^2+(y-y_o)^2)^2}i-2\frac{K_r(y-y_o)}{((x-x_o)^2+(y-y_o)^2)^2}j
\end{equation}
其方向矢量可以化简的表示为$[-(x-x_o)\; -(y-y_o)]$ 。因为斥力是个短程力,所以要设定一个阀值,使斥力小于这一阀值时为0。

在有了引力与斥力的公式后,就可以对每个点对车体的作用积分而算出合力,再通过合力的方向来计算机器人的行动方向。本机器人的控制系统所有的方向都表示为机器人前进方向与世界坐标系的y轴的夹角。算法的示意图如图\upcite{fig.pathPlaning1}所示。
\begin{figure}[!htp]
  \centering
  \includegraphics[width=0.5\textwidth]{chap4/pathPlaning1.PNG}
  \bicaption[fig.pathPlaning1]{虚拟势能蔽障算法示意图}{虚拟势能蔽障算法示意图\upcite{koren1991potential}}{Fig}{Motion of a mobile robot applying a potential field algorithm}
\end{figure}

但是这种算法存在的问题就是,如果障碍物足够大,可能存在着受力平衡点,使机器人永远无法越过障碍物。为了改变这一现象,不得不对机器人引入惯性。但是光引入惯性还是不够的,同时还要引入运动方向锁,来使最后的搜索过程可以得到收敛解。
\section{地图生成和最优路径生成}
关于地图的绘制,是受到了David Waller等人\upcite{waller1998transfer}研究的启发,同时参考了周斌等人\upcite{zb2006memory}的研究。底盘模块进行这样设计的原因是如果需要多次重复的使机器人从某一出发点到达某一目标点,在经过第一次机器人的探索后,可以记住环境信息,并在下次进行相同任务时,可以直接沿着最优路径前进而不需要重新探测。则可以大大加快机器人在同一环境长时间重复执行同一任务时的效率。

与周斌给出的方法不同,在本设计中,地图信息并不是保存在车体的控制器里,而是底盘模块会将数据发回给上位机,并由上位机处理并绘制地图,再通过上位机来计算最优路径,返回给片上系统。地图绘制的过程主要还是靠传感器模块标出障碍物的坐标,并根据这一坐标和机器人本身坐标尺寸来绘制障碍物和机器人可通过通道。

而最优路径生成这是通过现将绘制的地图网格化,然后从目标点出发向起始点回溯的去评价每一个点。评价公式就是与目标点的距离。在评价过程中只关心处在网格节点上的点,只要网格足够小,最后所得到的结果是可信的。当所有点评价完毕后,出发点会有多种评价值。找出最小的评价值,就可以找出最优的路径。然后连接这条线路上的每一个点,对这条路径的关键点信息进行计算并返回给底盘模块进行存储。这样就完成了最优路径生成的过程。

\section{本章小结}
本章主要介绍了所设计的移动模块化机器人系统的软件部分。这部分内容主要是针对轮式底盘和超声波传感器部分所设计的。本章节只涉及到了本移动式模块化平台的关键算法。关键算法包括4部分,第一为底盘的驱动算法,这是其他对车体控制算法的基础;其次是车体定位算法,这些算法是决定蔽障是否可行的关键;接下来是避障算法,详细介绍了机器人是如何能绕过障碍物而自主向目标点行进的;最后是地图的绘制和最优路径的生成,其中地图绘制是最优路径生成的基础。